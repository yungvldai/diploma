% Здесь выбираются необходимые графы, documentclass=eskdtext
\documentclass[
  utf8,
  14pt,
  russian,
  hpadding=10mm,
  vpadding=12mm,
  floatsection,
  columnxxxi,
  columnxxxii,
  columnxxvi,
  equationsection,
  pointsection,
  footnoteasterisk
]{eskdtext}

\usepackage{eskdstamp}
\usepackage{eskdfreesize}

\usepackage{fontspec}

% Что бы работал eskdx и некоторые другие пакеты LaTeX
\usepackage{xecyr}

% Межстрочный интервал - 1.5
\usepackage{setspace}
\onehalfspacing

% Отступ в начале абзаца - 1.25
\setlength{\parindent}{1.25cm}
\usepackage{indentfirst}

% кликабельные источники и содержание
% \usepackage{hyperref}
% заменить на 
\usepackage[pdfusetitle,hidelinks]{hyperref}
% в production сборке

% Списки
\usepackage{enumitem}
\usepackage{calc}% используется для сложения длин
\setlist[itemize]{%
  leftmargin=0pt, % согласно ГОСТ, вторая строка элемента списка должна
% начинаться без абзацного отступа.
% Отступ первой строки от левого края будет равен: абз. отступ + ' -- '.
% Хочу заметить, что это результат не совсем точный, но выглядит неплохо.
% (другого способа настроить нужный отступ не нашел. ¯\_(ツ)_/¯). TODO.
  itemindent={\the\parindent + \widthof{\ --\ }},
  itemsep=0cm, % лол, не знаю что это;
  nolistsep, % убираем большие скачки по вертикали;
  label=--% используем короткое тире вместо bullets.
}
\setlist[enumerate]{%
% Итоговый отступ элемента от левого края будет: 1.5cm + ширина ' 1) '.
% В результате на элементе из двух цифр, типа '10)', может вылезти за края 😨.
  leftmargin=0pt,
  itemindent={\the\parindent + \widthof{\ 1)\ }},
  itemsep=0cm,
  nolistsep,
  label={\arabic*)}%
}

% Настройка заголовков
\usepackage[raggedright,explicit]{titlesec}

\titleformat{name=\section}[block]
{\bfseries}%
{\hspace{1.25cm}\thesection}
{\widthof{\ \ }}
{\MakeUppercase{#1}}

\titleformat{name=\section,numberless}[block]
{\bfseries\centering}
{}
{0pt}
{\MakeUppercase{#1}}

\titleformat{\subsection}[block]
{\bfseries}% жирный шрифт
{\hspace{1.25cm}\thesubsection}% label
{\widthof{\ \ }}% два пробела между номером и названием подраздела
{#1}% before-code

\titlespacing{\section}{0pt}{18pt}{20pt}
\titlespacing{\subsection}{0pt}{18pt}{18pt}

% Для работы шрифтов
\usepackage{xunicode,xltxtra}

% Для работы с русскими текстами (расстановки переносов, последовательность комманд строго обязательна)
\usepackage{polyglossia}
\setdefaultlanguage{russian}

% Для того чтобы работали стандартные сочетания символов ---, --, << >> и т.п.
\defaultfontfeatures{Mapping=tex-text} 

% Шрифты
\setmainfont{Times New Roman}
\newfontfamily{\cyrillicfont}{Times New Roman}
\newfontfamily{\cyrillicfontrm}{Times New Roman}
\newfontfamily{\cyrillicfonttt}{Times New Roman}
\newfontfamily{\cyrillicfontsf}{Times New Roman}

% Отступы заголовков
\ESKDsectSkip{section}{7mm}{7mm}
\ESKDsectSkip{subsection}{5mm}{5mm}
\ESKDsectSkip{subsubsection}{3mm}{3mm}

% Для работы со сложными формулами
\usepackage{amsmath}
\usepackage{amssymb}

% Что бы использовать символ градуса (°) - \degree
\usepackage{gensymb}

% Для переноса составных слов
\XeTeXcharclass `\- 24
\XeTeXinterchartoks 24 0 ={\hskip\z@skip}
\XeTeXinterchartoks 0 24 ={\nobreak}

% Ставим подпись к рисункам. Вместо «рис. 1» будет «Рисунок 1»
\addto{\captionsrussian}{\renewcommand{\figurename}{Рисунок}}
% Убираем точки после цифр в заголовках
\def\russian@capsformat{%
  \def\postchapter{\@aftersepkern}%
  \def\postsection{\@aftersepkern}%
  \def\postsubsection{\@aftersepkern}%
  \def\postsubsubsection{\@aftersepkern}%
  \def\postparagraph{\@aftersepkern}%
  \def\postsubparagraph{\@aftersepkern}%
}

% Автоматически переносить на след. строку слова которые не убираются в строке
\sloppy

% Для вставки рисунков
\usepackage{graphicx}
% Объявляем поддерживаемые форматы
\DeclareGraphicsExtensions{.pdf,.png,.jpg,.jpeg}

% Настройка подписей: убираем пустую строку, ставим дефис как разделитель
\usepackage[labelsep=endash]{caption}

% Смещаем подпись к таблице влево
\captionsetup[table]{singlelinecheck=false,justification=raggedright}


% Используем расширенные таблицы (tabularx + longtable в одном)
\usepackage{ltablex}

% Выравниваем таблицы на всю ширину
\keepXColumns

% Для вставки интернет ссылок, полезно в библиографии
\usepackage{url}

% Подподразделы(\subsubsection) не выводим в содержании
\setcounter{tocdepth}{2}

% Каждый раздел (\section) с новой страницы
\let\stdsection\section
\renewcommand\section{\newpage\stdsection}

% Бибилиография
\usepackage[
  backend=biber,
  sorting=none,
  maxnames=50,
  style=gost-numeric
]{biblatex}

% улучшения 
\usepackage[all]{nowidow}
\usepackage{microtype}

% содержание
\makeatletter
\renewcommand{\l@section}{\@dottedtocline{1}{4ex}{4ex}}
\renewcommand{\l@subsection}{\@dottedtocline{1}{6ex}{6ex}}
\makeatother