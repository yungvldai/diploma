\section{СИНТЕЗ СИСТЕМЫ УПРАВЛЕНИЯ ДВИГАТЕЛЕМ}

\subsection{Линейная непрерывная модель}

Исходная модель содержит нелинейности и оказалась слишком сложной для синтеза регулятора для нее.
Поэтому было сделано предположение о том, что двигатель ДБМ в нашем случае работает
в режиме двигателя постоянного тока, а это значит, что динамику его движения можно
описать с помощью апериодического звена первого порядка с передаточной функцией \ref{eq:ap1}

\begin{equation}
  W(s)=\frac{k}{Ts+1},
  \label{eq:ap1}
\end{equation}

где $k$ -коэффициент передачи, $T$ - постоянная времени.

Следовательно, задача сводится к поиску таких $k$ и $T$, при которых переходные процессы
апериодического звена первого порядка и исходной модели будут иметь минимальные отличия.

\insertimage[scale=0.27]{react24}{
  Зависимость угловой скорости от времени при входном напряжении от блока питания
}

Была рассмотрена реакция исходной модели на входное воздействие равное 24 вольтам (рисунок \ref{react24}).
24 вольта используются по причине того, что блок питания стенда, как упоминалось равнее, выдает 
именно такое напряжение.

Установившееся значение~- 282.348~рад/с.
Коэффициент передачи $k$ был найден как отношение установившегося значения выхода модели ко входу (\ref{eq:count_k})

\begin{equation}
  k = \frac{\omega_{\text{уст}}}{U_{\text{вх}}} = \frac{282.348}{24} = 11.7645.
  \label{eq:count_k}
\end{equation}

Значение постоянной времени $T$ было найдено методом касательных к кривой переходного
процесса (рисунок \ref{gmt}).

\insertimage[scale=0.3]{gmt}{
  Нахождение постоянной времени
}

Затем значение $T$ было уточнено так, чтобы получить минимальную ошибку (рисунок \ref{lin_err}).
Итоговое значение постоянной времени $T = 0.0805$, а значит одним из возможных
математических описаний двигателя ДБМ является \ref{eq:lin_dbm}

\begin{equation}
  W(s)=\frac{11.7645}{0.0805s+1},
  \label{eq:lin_dbm}
\end{equation}

Сравнение исходного переходного процесса и полученного представлено на рисунке \ref{lin_comp};

\clearpage

\insertimage[scale=0.3]{lin_comp}{
  Сравнение ПП исходной и непрерывной линейной моделей
}

График ошибки линейной непрерывной модели относительно исходной представлен на рисунке \ref{lin_err}.

\insertimage[scale=0.3]{lin_err}{
  График ошибки
}

Можно заметить, что пиковое значение ошибки на графике ошибки (рисунок \ref{lin_err})
лежит около $-0.7$, это означает, что ошибка составляет
$\approx0.25\%$.

\subsection{Синтез регулятора угла положения ротора для линейной модели}

Пусть поставлена задача синтеза регулятора по углу со следующими показателями качества:

\begin{itemize}
  \item Перерегулирование $\sigma = 0\%$;
  \item Время переходного процесса $t_{\text{п}}=0.5$ с;
\end{itemize}

Обычно, в задачах регулирования угла важна точность позиционирования, поэтому для демонстрации
были выбраны именно такие показатели.

Выходом линейной непрерывной недели, представленной ранее, является угловая скорость $\omega$.
Для синтеза системы управления по углу был введено еще одно интегрирующее 
звено (рисунок \ref{syst}).

\insertimage[scale=0.7]{syst}{
  Структурная схема объекта управления
}

Передаточной функции интегрирующего звена соответствует дифференциальное уравнение \ref{eq:int_eq}

\begin{equation}
  \dot{x_1}=x_2.
  \label{eq:int_eq}
\end{equation}

Передаточной функции апериодического звена 1 порядка соответствует дифференциальное уравнение

$$
  0.0805\dot{x_2} + x_2 = 11.7645u,
$$

или, относительно $\dot{x_2}$

\begin{equation}
  \dot{x_2}=-\frac{1}{0.0805}x_2+146.1429u.
  \label{eq:ape_eq}
\end{equation}\\

Путем объединения уравнений \ref{eq:int_eq} и \ref{eq:ape_eq} в систему уравнений было получено
описание системы в форме вход-состояние-выход (ВСВ)

$$
  \begin{cases}
    \dot{X}=AX+Bu;
    \\
    Y=CX,
  \end{cases}
$$\\

где $X=\begin{bmatrix}x_1\\x_2\end{bmatrix}$, $A=\begin{bmatrix}0&1\\0& -12.4224\end{bmatrix}$, 
$B=\begin{bmatrix}0\\146.1429\end{bmatrix}$, $C=\begin{bmatrix}1&0\end{bmatrix}$.\\

Необходимыми условиями для синтеза алгоритмов управления методом модального управления
являются полная управляемость и полная наблюдаемость объекта управления \cite{МодальноеУправление}.

$$
  U = \begin{bmatrix}B&AB\end{bmatrix};
  \\
  Q = \begin{bmatrix}C\\CA\end{bmatrix},
$$

где $U$ - матрица управляемости, $Q$ - матрица наблюдаемости.

\begin{equation}
  rank(U) = rank(Q) = n,
  \label{eq:modal_check}
\end{equation}

где $n$ - порядок системы.

Выражение \ref{eq:modal_check} означает, что система годится для синтеза управления методом
модального управления.

В качестве регулятора был выбран ПИ-регулятор. Этот регулятор является частным случаем ПИД-регулятора
и довольно частно применяется на практике ввиду своих достоинств:

\begin{itemize}
  \item Обеспечивает нулевую статическую ошибку регулирования;
  \item Малая чувствительность к шумам в канале измерения (в отличие, например, от ПИД-регулятора);
  \item Простота настройки.
\end{itemize}

Задача синтеза ПИ-регулятора сводится к выбору матриц эталонной модели $\Gamma$ и $H$ и 
решению матричного уравнения Сильвестра (\ref{eq:sylv}) \cite{МодальноеУправление}
\begin{equation}
  M\Gamma-AM=-BH.
  \label{eq:sylv}
\end{equation}

И нахождению матрицы обратных связей с помощью выражения \ref{eq:bref}
\begin{equation}
  K = -HM^{-1}.
  \label{eq:bref}
\end{equation}

С учетом требуемого перерегулирования $<1\%$ в качестве эталонной модели был выбран 
полином Ньютона (биноминальный) 3 порядка (\ref{eq:newt}). Третий порядок
модели обусловлен наличием интегрирующего звена в регуляторе и вторым порядком
объекта управления.

\begin{equation}
  D^{\text{*}}(\lambda)=\lambda^3+3\omega_0\lambda^2+3\omega_0^2\lambda+\omega_0^3.
  \label{eq:newt}
\end{equation}

Затем был построен график переходного процесса (рисунок \ref{etalon_newt}) по полученной нормированной 
передаточной функции (\ref{eq:wsd}).

\begin{equation}
  \begin{cases}
    W(s)=\frac{1}{D^{\text{*}}(\lambda)};
    \\
    \lambda=s;
    \\
    \omega_0=1.
  \end{cases}
  \label{eq:wsd}
\end{equation}

Горизонтальными линиями на графике (рисунок \ref{etalon_newt}) обозначена $\Delta$-область, равная
$\pm5\%$ от установившегося значения ПП.

Временем переходного процесса или временем регулирования принято считать момент времени, когда
переходной процесс попадает в $\Delta$-область и больше ее не покидает. В случае полинома Ньютона
3 порядка время переходного процесса $t_{\text{п}}^{\text{*}}=6.316 c$.

\insertimage[scale=0.3]{etalon_newt}{
  Переходной процесс нормированной ПФ
}

Исходя из требуемого времени переходного процесса $\omega_0$ вычисляется как

$$
  \omega_0=\frac{t_{\text{п}}^{\text{*}}}{t_{\text{п}}}=\frac{6.316}{0.5}=12.632.
$$\\
Так, характеристический полином принимает вид

\begin{equation}
  D^{\text{*}}(\lambda)=\lambda^3+37.896\lambda^2+478.7023\lambda+2015.66.
  \label{eq:charpol}
\end{equation}
$$
  \lambda_{1,2,3}=-12.632.
$$

Все корни характеристического полинома (\ref{eq:charpol}) равны и вещественны, значит
матрицы эталонной модели на основе диагональной канонической формы 
задаются как:

$$
  \Gamma=\begin{bmatrix}-12.632&1&0\\0&-12.632&1\\0&0&-12.632\end{bmatrix}, H=\begin{bmatrix}1&0&0\end{bmatrix}.
$$

Можно заметить, что получившаяся матрица имеет размерность 3~на 3. Однако данные матрицы 
$A$ и $B$ имеют размерность 2~на 2. Если решить уравнение Сильвестра так, то неизвестная
матрица $M$ не будет квадратной, а значит мы не сможем посчитать обратную матрицу
для нахождения матрицы обратных связей $K$.

В \cite{МодальноеУправление} описывается расширение матриц $A$ и $B$ \ref{eq:matex} при
формировании расширенной модели ошибок в процессе
синтеза астатического (ПИ-) регулятора.

\begin{equation}
  \overline{A}=\begin{bmatrix}0&C\\0&A\end{bmatrix}, \overline{B}=\begin{bmatrix}0\\B\end{bmatrix}.
  \label{eq:matex}
\end{equation}

Затем было решено матричное уравнение Сильвестра следующего вида:
$$
  M\Gamma-\overline{A}M=-\overline{B}H.
$$

В результате получена матрица $M$:

\begin{equation}
  M=\begin{bmatrix}-5.1831&-30.0071&-169.0569\\65.3076&372.9065&2100.1103\\-822.8767&-4633.3147&-26088.4836\end{bmatrix}.
  \label{eq:m}
\end{equation}

Полученный результат \ref{eq:m} был подставлен в \ref{eq:k}

\begin{equation}
  K=-HM^{-1}=\begin{bmatrix}13.6878&3.259&0.1736\end{bmatrix}.
  \label{eq:k}
\end{equation}

После получения матрицы обратных связей была проведена проверка равенства
собственных чисел матрицы замкнутой системы $F = A - BK$ и матрицы $G$.
Проверка подтвердила правильность расчета.

Структурная схема регулятора с учетом отрицательной обратной связи представлена на рисунке \ref{reg_demo}.

\insertimage[scale=0.5]{reg_demo}{
  Структурная схема системы управления
}

Путем подачи входного воздействия равному $U=1$ был получен график переходного процесса \ref{reg_demo_chart}.

\insertimage[scale=0.3]{reg_demo_chart}{
  График переходного процесса
}

Затем был совершен переход к более реальным условиям. Теперь в качестве объекта управления
выступает нелинейная модель двигателя, построенная ранее, а требуемый угол 
регулирования~- 50~радиан (рисунок \ref{demo_full}).

\insertimage[scale=0.55]{demo_full}{
  Структурная схема системы управления
}

Результат моделирования представлен на рисунке \ref{demo_full_chart}.

\clearpage

\insertimage[scale=0.3]{demo_full_chart}{
  График переходного процесса
}

В итоге время переходного процесса составило $\approx0.778$ с, а перерегулирование $\approx0.16\%$,
что можно считать вполне приемлемым результатом.