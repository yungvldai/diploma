\conclusion

В ходе проделанной работы был разработан учебный стенд.
Разработанный стенд отвечает всем требованиями, установленным в 
техническом задании.

Питание силовой части стенда осуществляется от бытовой сети 220~вольт.
Вычислительная часть стенда питается либо от USB, либо через \mbox{DB-25} при
подключении к компьютеру.

Стенд спроектирован модульно. Это означает, что он будет работать
даже если нагрузочный двигатель вовсе не подключен. Например, в задаче
исследования скоростных характеристик двигателя ДБМ без внешней нагрузки. Это
работает и в обратную сторону - благодаря способу подключения отладочной
платы к стенду пользователь имеет доступ ко всем ножкам отладочной платы
и может использовать их для подключения любых других дополнительных устройств.

Стенд позволяет собирать различные данные, которые могут пригодиться для
управления, например, угол положения ротора. В таблице \ref{tab:stend_sens} 
представлены данные о разрешающей способности датчиков стенда.

\begin{tabularx}{\textwidth}{|X|c|}
  \caption{Разрешающая способность датчиков}\label{tab:stend_sens}\\
  \hline
    Разрешение для измерения угла поворота ротора, рад & 0.000383 \\
  \hline
    Разрешение для измерения тока в обмотках, А & 0.000983 \\
  \hline
    Разрешение для измерения напряжения в обмотках, В & 0.00586 \\
  \hline
\end{tabularx}

Проведенный синтез регулятора показал, что ожидаемое время переходного 
процесса примерно 0.8 с. Такое время позволяет наглядно наблюдать 
переходной процесс.

Разработанный стенд позволяет реализовать и исследовать системы управления 
с различными режимами управления двигателем~- от простых шаговых до 
современного векторного управления. Таким образом, он хорошо подходит для
обучения студентов принципам и тонкостям управления бесколлекторными двигателями.

Подсчет полной стоимости стенда представлен в таблице \ref{tab:total}
(все цены в рублях).

\begin{tabularx}{\textwidth}{|X|c|}
  \caption{Подсчет полной стоимости стенда}\label{tab:total}\\
  \hline
    Двигатель ДБМ-63 & 9 500 \\
  \hline
    Изготовление печатной платы & 3 600 \\
  \hline
    Электронные компоненты & 7 600 \\
  \hline
    Блок питания NES-100-24 & 1 200 \\
  \hline
    Двигатель Д5-ТР & $\approx1\text{ }000$ \\
  \hline
    Расходники и отдельные детали & $\approx1\text{ }000$ \\
  \hline
    \textbf{Итого} & \textbf{23900} \\
  \hline
\end{tabularx}

Под расходниками подразумевается припой, флюс, различного вида
крепежные изделия и так далее.