\conclusion

В заключение хотелось бы рассказать о достигнутых результатах.
В рамках выпускной квалификационной работы был разработан 
стенд для управления бесколлекторным электродвигателем.
На рисунке \ref{done} представлен электронный модуль стенда.

\insertimage[scale=0.25]{done}{
  Разработанный электронный модуль
}

Данный модуль поддерживает подключение до 4 обмоток двигателя, измерение тока,
напряжения в каждой обмотке, измерение угла положения ротора, отправку данных на компьютер.

Двигатель ДБМ, помещенный в стакан (рисунок \ref{stakan}), подключается к электронному
модулю с помощью восьми клеммников, затем на торец устанавливается 
печатная плата с датчиком угла положения ротора и стенд готов к работе.

Опционально также можно установить стакан с нагрузочным двигателем Д5-ТР, соединив
валы муфтой и подключив установку к электронному модулю.

\insertimage[scale=0.2]{stakan}{
  Двигатель ДБМ в стакане
}