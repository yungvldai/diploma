\intro

В современном мире электропривод — это, без преувеличения, одна из важнейших частей 
систем автоматизации производственных процессов. Основной задачей конструкторов
является проектирование и реализация электроприводов с как можно большими скоростными
и точностными характеристиками. 

В наши дни все большую популярность набирают появившиеся 
относительно недавно бесконтактные моментные электродвигатели. 
Главное отличие таких двигателей от обычных ДПТ — это отсутствие корпуса, 
вала, подшипников и, конечно, коллекторов. Такие двигатели предназначены для 
встраивания в объект управления без редуктора. Это важно, потому что редуктор — 
это дорогостоящий и шумный узел, но еще важнее то, что он отрицательно влияет 
на точность управляемого электропривода из-за наличия зазоров и 
упругих деформаций. Также немаловажно, что отсутствие коллекторных щеток исключает явление искрения
этих самых щеток и значительно повышает надежность и срок эксплуатации электромашины, а
также допускает применение таких двигателей, например, на взрыво-/огнеопасных производствах.

Двигатели такого типа предназначены для работы в локально замкнутой (с датчиками положения ротора) 
или разомкнутой по углу системах управления и находят широкое применение в:

\begin{itemize}
  \item системах автоматического управления, работающих в особо 
    тяжелых условиях эксплуатации;

  \item быстродействующих следящих системах управления высокой точности;

  \item медицинском приборостроении по причине того, что к технике в медицинской сфере
    предъявляются высокие требования к уровню шума, уровню пульсаций 
    вращающего момента и другим подобным характеристикам;
  
  \item бытовых товарах, например, стиральных машинах;
  
  \item исполнительных робототехнических системах автоматического управления;
  
  \item военной промышленности.
\end{itemize}

На сайте \cite{Машиноаппарат} разработчиков и производителей двигателей серии ДБМ ОАО <<МАШИНОАППАРАТ>> 
представлены конкретные примеры применения, вот лишь некоторые из них:

\begin{itemize}
  \item Оптико-локационная станция ОЛС-УЭ для самолетов-истребителей;

  \item Панорамический прицел командира, устанавливаемый на башню танка;

  \item Солнечные датчики 331К, использующиеся на спутниках системы ГЛОНАСС.
\end{itemize}

Такие сферы применения обусловлены тем, что моментные двигатели обеспечивают
высокие вращающие моменты на небольшой скорости и одновременно высокую 
повторяемость, динамику и точность позиционирования. 

Очевидно, что для управления приводами, построенными на бесконтактных моментных
электродвигателях необходимы несколько иные системы управления и, как следствие, 
схемотехчнические и программные решения. Разработанный макет нужен как раз таки для
решения исследовательских и образовательных задач.