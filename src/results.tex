\section{РЕЗУЛЬТАТЫ МОДЕЛИРОВАНИЯ}

Как видно из рисунков \ref{demo_full_chart} и \ref{reg_demo_chart}, регулятор, 
синтезированный
для линейной непрерывной модели, работает для обеих моделей и 
обеспечивает регулирование системы к заданному углу. Хотя, сравнивая
переходные процессы линейной непрерывной модели и исходной
наблюдается некоторое отклонение (рисунок \ref{mod_err}), а также 
небольшое отклонение от заданных показателей качества.

\insertimage[scale=0.3]{mod_err}{
  Сравнение двух объектов управления
}

А на рисунке \ref{mod_err2} представлена ошибка
регулирования нелинейной системы относительно входного сигнала.

\insertimage[scale=0.3]{mod_err2}{
  Ошибка регулирования нелинейной системы управления
}