\section{РАЗРАБОТКА ФУНКЦИОНАЛЬНОЙ СХЕМЫ СТЕНДА}

При разработке функциональной схемы главной задачей было разъяснить и описать
процессы, протекающие как между отдельными цепями стенда, так на стенде в целом.

Главным компонентом стенда стенда является исследуемый двигатель~- двигатель ДБМ. Двигатель нужно
обеспечить питанием, поэтому на стенде предусмотрена установка покупного блока питания.
Однако, сигналы на управление двигателем не могут работать с таким напряжение, а
пины микроконтроллера просто не выдержат протекающего тока. Отсюда формулируется
задача разработки специального драйвера, который с помощью силовых ключей позволял бы
осуществлять управление двигателем.

В качестве объекта изучения предлагается осуществлять сбор и обработку различных метрик, таких как:
\begin{itemize}
  \item положение ротора;
  \item токи на обмотках;
  \item напряжения на обмотках;
  \item момент на валу.
\end{itemize}

Это стало возможным благодаря установке определенных датчиков. Каждый из этих датчиков
должен осуществлять отправку данных на какое-либо управляющее устройство. И в качестве такого
устройства было решено использовать микроконтроллер. В таком случае микроконтроллер сможет
либо сам реализовывать алгоритм управления, либо же просто собирать данные и в формате 
единого пакета отправлять их на компьютер.

Микроконтроллер должен быть достаточно мощным, так как известны случаи, когда вычислительных
мощностей для управления такими двигателями просто не хватало [можно вставить источник]. 
Микроконтроллер должен поддерживать соединение с компьютером для отправки показаний или же
приема команд управления.
При этом на микроконтроллер также ложится еще одна немаловажная задача~- управление драйвером
для двигателя.

Описанной выше конфигурации хватает для изучения скоростных и точностных характеристик исследуемого
двигателя. Тем не менее, для изучения силовых параметров был добавлен еще один двигатель
для создания искусственной нагрузки на валу. Этот двигатель также управляется
с помощью микроконтроллера через драйвер. Было решено не реализовывать возможность реверса этого
двигателя, потому что этот функционал просто не нужен. 

Данные о моменте не валу предлагается собирать с помощью тензодатчика. Более подробно
данное решение будет описано в следующей главе.

Исходя из изложенных выше требований была разработана функциональная схема (приложение ФСУиР.205.R3435.001 Э2). 