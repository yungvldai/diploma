\section{ОБЗОР СУЩЕСТВУЮЩИХ УЧЕБНЫХ СТЕНДОВ НА БАЗЕ БЕСКОЛЛЕКТОРНЫХ ДВИГАТЕЛЕЙ}

\subsection{Типовой комплект учебного оборудования <<Вентильный двигатель>>}

\insertimage[scale=0.25]{stend1}{
  Внешний вид комплекта
}

Стенд (рисунок \ref{stend1}) предназначен для разработки и исследований новых двигательных 
установок транспортных и подъемно-перегрузочных систем, основанных на 
использовании бесколлекторного двигателя постоянного тока с постоянными 
магнитами, систем управления такими установками, с использованием 
датчиков Холла и различными способами коммутации обмоток двигателя, а 
также особенностей их конструкции и электромеханических узлов и агрегатов, 
рабочих электромеханических процессов. 

Комплект представляет из себя полный набор всего, что нужно:
двигатель для исследования, нагрузочный двигатель, осцилограф. В комплектацию продажи
может быть добавлен даже стол. 

Одним из главных недостатков данного стенда является то, что двигатель спрятан от глаз пользователя.
Задача обучения включает в себя не только познание теории, но и получение практических навыков.
Фактически же реальное изучаемое устройство просто скрыто. Также, стенд не предлагает никаких
возможностей дальнейшей работы с полученными данными. Дальше осцилографа эти данные никуда не идут,
а значит носят исключительно демонстрационный характер.
Управление стендом осуществляется с помощью набора тумблеров, реостатов и других подобных
электронных компонентов. Соответственно, о реализации какой-то сложной программной 
системы управления на таком стенде не может идти и речи. Ну и, конечно, стоимость данного продукта,
на момент апреля 2021 года она составила 334~490 рублей за настольное исполнение и 368~810 рублей за
стационарное исполнение со столом.

\subsection{Стенд SkyRC Extreme BMC-01 для проверки бесколлекторных двигателей}

\insertimage[scale=0.45]{stend3}{
  Внешний вид прибора
}

Этот прибор является высокоточным электронное устройством специально разработанное для 
проверки бесколлекторных электромоторов. Он может измерять такие значения как:

\begin{itemize}
  \item обороты в минуту;
  \item обороты на вольт;
  \item ток,
\end{itemize}

а также, проверять функционирование датчика Холла (для датчиковых моторов). 
Стенд оснащён жидкокристаллическим дисплеем 2x16 знаков, который отображает в 
режиме реального времени измерения значений.

Прибор поддерживает работу с датчиковыми и бездатчиковыми бесколлекторными моторами.

В комплект поставки сам двигатель не входит. Более существенный минус заключается в том, что 
снятые данные сложно обрабатывать. Такой модуль
годится только для простых лабораторных работ, просто ради ознакомления с изучаемой темой. Зато, 
он имеет относительно невысокую цену~- 10~562 рублей на момент апреля 2021 года.

\subsection{
  Типовой комплект учебного оборудования <<Микропроцессорная система управления вентильным двигателем", исполнение моноблочное с ноутбуком>>
}

\insertimage[scale=0.65]{stend2}{
  Внешний вид комплекта
}

Лабораторный стенд представляет собой моноблок, в котором реализована микропроцессорная 
система управления вентильным электродвигателем. Функционально стенд состоит из двух 
частей – бесколлекторный двигатель и микроконтроллер AVR.

Из недостатков этого комплекта можно отметить, опять же, выскокую цену~- 111~804 рублей на 
момент апреля 2021 года, а также выбор микроконтроллера. Стенд выполнен на базе 
микроконтроллера Atmega~8535, который является восьмибитным, имеет всего 8 килобайт 
флеш-памяти и подходит не для всех вычислительных задач.